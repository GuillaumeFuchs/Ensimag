\documentclass[french,12pt,a4paper]{article}
\usepackage[T1]{fontenc}
\usepackage[utf8]{inputenc}
\usepackage[dvips]{graphicx}
%\usepackage[english]{babel}
\usepackage[frenchb]{babel}
\AddThinSpaceBeforeFootnotes % à insérer si on utilise \usepackage[french]{babel}
\FrenchFootnotes % à insérer si on utilise \usepackage[french]{babel}
\usepackage{amsmath,amsthm,amsfonts,amssymb}
\usepackage{mathrsfs}
\usepackage{array}
\usepackage{color}
\usepackage{float}
\usepackage{pstricks,pstricks-add,pst-plot,pst-tree}
\usepackage{enumerate}
\usepackage{textcomp}
\usepackage{setspace}
\usepackage{lettrine}
\usepackage{lscape}
\usepackage{lmodern}
\usepackage{stmaryrd}
\usepackage{subfigure}
\usepackage{multido}
\usepackage{listings}
\usepackage{footnote}
\usepackage{appendix}
\usepackage{color}
\usepackage{listings}
\usepackage{tikz}
\usepackage{dsfont}
\usetikzlibrary{matrix}

\lstset{
  language={C++},
  numbers=left, numberstyle=\tiny, stepnumber=1, firstnumber=last,
  frameround=tttt, 
  frame=single, 
  float,
  captionpos=b,
  breaklines=true,
  sensitive=f,
  morestring=[d]",
  basicstyle=\small\ttfamily,
  keywordstyle=\bf\small,
  stringstyle=\sf
}
\usepackage{fancyhdr,lastpage}
\usepackage[twoside,left=2cm,top=2.5cm,dvips,marginparwidth=1.9cm,marginparsep=0.5cm,headheight=35pt]{geometry}
%En tete et pied de page
\usepackage{fancyhdr}
\pagestyle{fancy}
\lhead{\leftmark} 
\chead{}
\rhead{PEPS}
\lfoot{ENSIMAG 3A}
\cfoot{\textit{Equipe 7}}
\rfoot{\thepage}
\renewcommand{\headrulewidth}{0pt}  
\renewcommand{\footrulewidth}{0.4pt}
\title{Projet .NET : Gestion indicielle}
\date{October 31, 475}
\author{Guillaume Fuchs, Guillaume Pelletier, Samuel Rosilio}
%Page de garde

\begin{document}
\begin{titlepage}
\begin{center}

\textsc{\LARGE ENSIMAG}\\[1.5cm]

\textsc{\Large Projet d'évaluation de produit structuré}\\[0.5cm]

% Title
 \hrule
 \hrule 

\vspace{7mm}
{ \huge \bfseries Playlist 2  }

\vspace{7mm}
\hrule
\hrule

\vspace{7mm}
% Author and supervisor
\begin{minipage}{0.4\textwidth}
\begin{flushleft} \large
\emph{Etudiants:}\\
Guillaume \textsc{Fuchs},\\
Guillaume \textsc{Pelletier},\\
Samuel \textsc{Rosilio}
\end{flushleft}
\end{minipage}

\vfill

% Bottom of the page
{\large \today}

\end{center}
\end{titlepage}
\tableofcontents
\newpage

\section{Introduction}

Ce rapport est l'objet d'une analyse du produit structuré nommé \lstinline!Playlist 2! déposé par  la CACEIS Bank, commercialisé par le réseau des caisses d'épargne et géré par Natixis Asset Management.
Ce produit est un OPCVM français qui pouvait être acheté entre le 14 janvier 2010 et le 28 juin 2010 avec pour valeur nominale liquidative de référence la valeur le maximale parmi toutes les valeurs liquidatives entre le 14 janvier et le 28 juin 2010. L'objectif de gestion du FCP est de permettre au porteur de recevoir la valeur liquidative de référence majorée d'un gain final à une date donnée dépendant des performances de 4 indices boursiers. 


\section{Définition}

Le \textbf{dépositaire} du produit financier est l'organisme qui sera chargé de :\\
\begin{itemize}
\item[•]
	La garde des avoirs en dépôt et leur restitution.
\item[•]
	Le dépouillement des ordres.
\item[•]
	L'information de la société de gestion ou de la SICAV des opérations relatives aux titres conservés pour son compte.
\item[•]
	Le contrôle de la régularité des décisions de l'OPC ou de sa société de gestion par rapport aux dispositions législatives et réglementaires applicables.\\
\end{itemize}


La \textbf{société de gestion} associée au produit devra s'occuper de :\\
\begin{itemize}
\item[•]
	La gestion de portefeuille pour le compte de tiers ce qui consiste à gérer des portefeuilles individuels d'instruments financiers pour le compte de clients, qu'il s'agisse par exemple de clients particuliers ou d'investisseurs institutionnels. Un mandat de gestion est conclu entre la société de gestion et son client.
\item[•]
	La gestion collective ou gestion d’organismes de placement collectif (OPC) consiste schématiquement à gérer des portefeuilles collectifs. Un OPC est constitué des sommes mises en commun par des investisseurs et gérées pour leur compte par un gestionnaire de portefeuille. Ce dernier utilise ces sommes pour acquérir des instruments financiers, par exemple des actions ou des obligations en fonction de ses objectifs. Des parts ou des actions représentant une quote-part de l’avoir de l’OPC sont émises, en contrepartie des sommes versées dans l’OPC. \\
\end{itemize}


Un FCP fait parti de la famille des OPCVM.\\
\begin{itemize}


\item[•]
Un \textbf{OPCVM} est une entité qui gère un portefeuille dont les fonds sont placés en valeurs mobilières. Garantie de l’investissement.\\

\item[•]
Une \textbf{OAT} est une obligation assimilable du Trésor, emprunts d’Etat émis pour une durée de 5 à 50 ans.\\

\item[•]
Un \textbf{fonds à formule} regroupe plusieurs catégories d’OPCVM. Il offre une perspective de gain dépendant des évolutions des marchés financiers, selon des paramètres définis à la souscription. Il offre une garantie sur le capital initialement investi. Il s'occupe de la gestion active du fonds dans des actifs non-risqués (monétaires, obligataires …) pour garantir le capital et des actifs risqués (actions, dérivées …) à la recherche de performance.\\

\item[•]
Une \textbf{OCDE} est une organisation de coopération et de développement économique. Elle publie des études statistiques. Les membres ont un système de gouvernement démocratique et une économie de marché.\\

\item[•]
Une \textbf{caisse d’amortissement de la dette sociale} est un organisme gouvernemental français qui possède de la dette sociale.\\

\item[•]
La \textbf{BPCE} est l'ensemble des entreprises qui composent la Caisse nationale des Caisses d’épargne et de la Banque fédérale des Banques populaires.\\

\item[•]
Le \textbf{dépositaire} d'un OPCVM est un établissement qui assure la conservation des actifs et le contrôle de la régularité des décisions de l'OPCVM.\\

\item[•]
La \textbf{société de gestion} d'un OPCVM est chargée de la gestion administrative, comptable et financière de l'OPCVM.\\

\item[•]
Le \textbf{FCP de capitalisation des revenus} représente tous les revenus investis dans le portefeuille du fonds propre.\\

\item[•]
Un \textbf{indice boursier} est une mesure statistique calculée par le regroupement des valeurs de titres de plusieurs sociétés.\\
Attention les indices des grands marchés mondiaux sont de plus en plus corrélés entre eux.\\

\item[•]
La \textbf{valeur liquidative} est la division de l’actif net de l’OPCVM par son nombre de parts calculée toutes les 2 semaines. Ceci implique des actifs d'une valeur inférieure à 80 millions €.\\

\item[•]
La \textbf{date de clôture de l'exercice} est la date à laquelle les résultats de l'année sont calculés.\\

\item[•]
La \textbf{performance} est la plus ou moins-value réalisée par rapport à l’investissement initial.\\

\item[•]
La \textbf{performance avec dividendes réinvestis} consiste en le fait que les dividendes sont réinvestis le jour même pour souscrire des actions supplémentaires. Cela permet d’estimer l’évolution véritable de la valeur de l’OPCVM, indépendamment de son mode de distribution. \\

\item[•]
Une \textbf{cession temporaire des titres} consiste en la vente de titres contre espèces ou autres titres avec un engagement irrévocable de part et d’autres de restituer les valeurs échangées.\\

\item[•]
Un \textbf{titre garanti} est un titre comportant un garant.\\

\item[•]
Un \textbf{Swap de performance} consiste à échanger de la performance d'une action contre un taux d'intérêt.\\

\item[•]
Une \textbf{exposition} (financière) est une limitation financière.\\

\item[•]
Un \textbf{Warrant} est un produit boursier à effet de levier qui donne le droit d'acheter ou de vendre à un prix donné un produit financier.\\

\item[•]
Un \textbf{Asset Swap} en terme général est un Swap sur actif.\\

\item[•]
Un \textbf{Swap} est un contrat d'échange de flux financiers entre deux parties jusqu'à une maturité et à fréquence déterminée.\\

\item[•]
Un \textbf{Cap} est un produit financier visant à assurer par son achat un niveau maximal sur un indice de taux révisable tout en profitant d'une éventuelle stabilité ou baisse de ce taux révisable. L'acheteur de Cap paie une prime au vendeur de Cap.\\

\item[•]
Un \textbf{Floor} est un produit financier visant à assurer par son achat un niveau minimal sur un indice de taux révisable tout en profitant d'une éventuelle stabilité ou hausse de ce taux révisable. L'acheteur d'un Floor paie une prime au vendeur du Floor.\\

\item[•]
Un \textbf{Collar} consiste en l'achat d'un Cap et la vente d'un Floor. L'objectif est de réduire le coût de la couverture contre le risque de taux.\\

\item[•]
Un \textbf{FRA} est un produit dérivé du marché monétaire. Il s'agit d'un contrat Forward, négocié sur le marché OTC (marché de gré à gré), entre deux contreparties et dont l'objectif est la fixation dès aujourd'hui d'un taux in fine de référence convenu sur un principal donné, pendant une période future spécifiée.\\

\item[•]
Les \textbf{titres de taux} regroupent l'ensemble des obligations à taux fixe et à taux variable. \\

\item[•]
Un \textbf{emprunt d'espèces} est un emprunt de liquidité sous forme de trésorerie. A comparer avec l'emprunt de titres.\\

\item[•]
Un \textbf{dépôt} est une opération de prêt qui génère une créance dans le patrimoine de l'OPCVM et une dette pour la banque. \\

\item[•]
La \textbf{prise et mise en pension} (cession de titre) est relative au taux REPO qui désigne une transaction dans laquelle deux parties s'entendent simultanément sur deux transactions: une vente de titres au comptant suivie d'un rachat à terme à une date et à un prix convenu d'avance. Cette transaction est qualifiée de pension livrée (prise en pension des titres par le prêteur de cash et mise en pension des titres par le prêteur de titres).\\

\item[•]
Le \textbf{titre de créance négociable} est un instrument financier qui suivant la nature de l'émetteur peut être: un Bon du Trésor à taux fixe (titre à court terme émis par le Trésor), un Bon du Trésor à intérêts annuels (titre moyen terme émis par le Trésor), un billet de trésorerie émis par les entreprises, un certificat de dépôts émis par les banques, un bon à moyen terme négociable émis par les entreprises et les établissements de crédit, un bon des institutions financières spécialisées émis par certains établissements du secteur financier public ou para-public.\\

\end{itemize}

\newpage
\section{Analyse des flux financiers du produit}

\subsection{Description du produit}

Notre produit financier prend appui sur l'évolution de 4 indices boursiers à travers le monde, les flux versés dépendront en effet des performances de chacun de ces indices depuis leur niveau d'origines (moyenne arithmétique des niveaux de clôture publiés le 29 avril 2010, 30 avril 2010 et le 3 mai 2010).
Les quatre indices considérés sont le Footsie 100, le Standard \& Poor's 500, Dow Jones Euro Stoxx 50 et le Nikkei 225.\\
\indent Le fonctionnement de notre produit est alors le suivant, si le 25 avril 2011 au moins trois des quatre indices ont eu un rendement supérieur à 10\%  alors un flux d'au moins 104,5\% de la valeur nominale nous sera versé à échéance. De plus si trois des quatre indices ont eu un rendement supérieur à 20\% alors la date d'échéance devient le 2 avril 2011 et nous recevons 104,5\% de la valeur nominale.\\
\indent En date du 23 avril 2012, nous raisonnons de la même façon en effet si trois des quatre indices ont eu un rendement supérieur à 10\% par rapport à leur niveau d'origine alors on obtiendra 4,5\% de plus que ce que l'on devrait déjà acquérir. En revanche si trois des quatre indices ont dépassé 20\% de rendement, la date d'échéance devient le 30 avril 2012.\\
Pour ce qui est des années 2013, 2014, 2015 et 2016, on accumule simplement 4,5\% en plus à échéance si trois des quatre indices ont un rendement supérieur à 10\% aux dates de constations. La date d'échéance finale (si aucun arrêt de la formule en 2011 et 2012) est le 25 avril 2016.\\

\subsection{Flux envisageables du produit}

\noindent Commençons par envisager les différents cas possibles :\\
Soit 
\[ A = \{ \text{Dow Jones Euro Stoxx 50}, \text{Standard \& Poor's 500}, \text{Footsie 100}, \text{Nikkei 225} \} \]\\
Ainsi 
\[ \forall \text{i, j, k, l} \in \text{A tels que A} = \{ \text{i, j, k, l} \} \]\\

\begin{itemize}

\item[•]
Cas où l'échéance est en fin de première année :\\

\begin{spacing}{1.2}
\begin{center}
\begin{tabular}{|c|c|c|c|c|c|}
  \hline
  Date & $R_{i}$ & $R_{j}$ & $R_{k}$ & $R_{l}$ & Prime accumulée \\
  \hline
  25/04/2011 & 21\% & 15.6\% & 22.6\% & 31.1\% & 4.5\%\\
  \hline
\end{tabular}
\end{center}
\end{spacing}
\indent \\
\indent \\
On constate dès la première année que trois des quatre indices dépassent un rendement de 20\% ainsi le produit arrive à échéance en fin de première année et le client recevra 104,5\% puisque par conséquent trois des quatre indices ont eu une performance supérieure à 10\% \\

\item[•]
Cas le moins intéressant pour le client avec échéance deux ans :\\

\begin{spacing}{1.2}
\begin{center}
\begin{tabular}{|c|c|c|c|c|c|}
  \hline
  Date & $R_{i}$ & $R_{j}$ & $R_{k}$ & $R_{l}$ & Prime accumulée \\
  \hline
  25/04/2011 & 5.8\% & 15.6\% & 2.6\% & -7.1\% & 0\%\\
  23/04/2012 & 21.3\% & 28.8\% & 20.4\% & 10.3\% & 4.5\%\\
  \hline
\end{tabular}
\end{center}
\end{spacing}
\indent \\
\indent \\
Dans ce cas ci, aucune prime n'est acquise grâce aux performances des indices la première année, de plus la seconde année comme trois des quatre performances dépassent 20\%, l'échéance devient le 30/04/2012 et 104,5\% sera versé au client.\\

\item[•]
Cas le plus intéressant pour le client avec échéance deux ans :\\
\begin{spacing}{1.2}
\begin{center}
\begin{tabular}{|c|c|c|c|c|c|}
  \hline
  Date & $R_{i}$ & $R_{j}$ & $R_{k}$ & $R_{l}$ & Prime accumulée \\
  \hline
  25/04/2011 & 13.8\% & 15.6\% & 10.6\% & -7.1\% & 4.5\% \\
  23/04/2012 & 21.3\% & 28.8\% & 20.4\% & 5.3\% & 9\% \\
  \hline
\end{tabular}
\end{center}
\end{spacing}
\indent \\
\indent \\
Ainsi dans ce cas le client recevra une prime de 9\% de la valeur liquidative de référence en fin de deuxième année. L'échéance devient donc le 30/04/2012 et 109\% de la valeur liquidative de référence sera versé au client.\\

\item[•]
Nous allons maintenant distinguer les deux cas extrêmes dans le cas d'une échéance à six ans, le premier cas correspond à celui où le client ne gagnera rien soit :\\
\begin{spacing}{1.2}
\begin{center}
\begin{tabular}{|c|c|c|c|c|c|}
  \hline
  Date & $R_{i}$ & $R_{j}$ & $R_{k}$ & $R_{l}$ & Prime accumulée \\
  \hline
  25/04/2011 & 3.8\% & 5.6\% & 0.6\% & -7.1\% & 0\% \\
  23/04/2012 & 10.3\% & 8.8\% & -10.4\% & 5.3\% & 0\% \\
  29/04/2013 & 4.5\% & 12.4\% & 4.2\% & 14.2\% & 0\%\\
  29/04/2014 & -5.4\% & 16.3\% & 10.2\% & 9.4\% & 0\%\\
  29/04/2015 & 3.3\% & 11.1\% & 7.1\% & 18.2\% & 0\%\\
  25/04/2016 & 11.5\% & 15.2\% & 8.7\% & 9.9\% & 0\%\\
  \hline
\end{tabular}
\end{center}
\end{spacing}
\indent \\
\indent \\
Dans ce cas ci qui est le cas le plus défavorable sur un horizon de six ans, le client retouchera la valeur liquidative de référence à la fin de la sixième année car trois des quatre indices n'ont jamais eu une performance supérieure à 10\%.\\

\newpage
\item[•]
Cas le plus favorable pour une échéance de six ans :\\
\begin{spacing}{1.2}
\begin{center}
\begin{tabular}{|c|c|c|c|c|c|}
  \hline
  Date & $R_{i}$ & $R_{j}$ & $R_{k}$ & $R_{l}$ & Prime accumulée \\
  \hline
  25/04/2011 & 13.8\% & 15.6\% & 10.6\% & -7.1\% & 4.5\% \\
  23/04/2012 & 10.3\% & 18.8\% & -10.4\% & 15.3\% & 9\% \\
  29/04/2013 & 14.5\% & 12.4\% & 4.2\% & 14.2\% & 13.5\%\\
  29/04/2014 & -5.4\% & 16.3\% & 10.2\% & 19.4\% & 18\%\\
  29/04/2015 & 3.3\% & 11.1\% & 17.1\% & 18.2\% & 22.5\%\\
  25/04/2016 & 21.5\% & 15.2\% & 18.7\% & 29.9\% & 27\%\\
  \hline
\end{tabular}
\end{center}
\end{spacing}
\indent \\
\indent \\
Dans ce cas ci, le client touchera à la fin de la sixième année 127\% de la valeur liquidative de référence puisque chaque année, au moins trois des quatre indices ont une performance supérieure à 10\%.\\
\end{itemize}

\noindent On peut résumer avec l'arbre suivant les flux possibles à échéance du produit (en notant 0 les nœuds non terminaux)  :

% Set the overall layout of the tree
\tikzstyle{level 1}=[level distance=4cm, sibling distance=3cm,->]
\tikzstyle{level 2}=[level distance=4cm, sibling distance=2cm,->]

% Define styles for bags and leafs
\tikzstyle{bag} = [text width=2em, text centered]
\tikzstyle{end} = []

% The sloped option gives rotated edge labels. Personally
% I find sloped labels a bit difficult to read. Remove the sloped options
% to get horizontal labels. 
\begin{tikzpicture}[>=stealth,sloped]
    \matrix (tree) [%
      matrix of nodes,
      minimum size=1cm,
      column sep=3cm,
      row sep=0.9cm,
    ]
    {
          &       &      & 190.5\\
          &       & 163.5  & 183.75\\
          & 156.75 &      & 177 \\
      150 &       & 156.75& 170.25 \\
          & 0   &      & 163.5 \\
          &       & 0  & 156.75\\
          &       &      & 150 \\
    };
    \draw[->] (tree-4-1) -- (tree-3-2) node [midway,above] {};
    \draw[->] (tree-4-1) -- (tree-5-2) node [midway,below] {};
    \draw[->] (tree-3-2) -- (tree-2-3) node [midway,above] {};
    \draw[->] (tree-3-2) -- (tree-4-3) node [midway,below] {};
    \draw[->] (tree-5-2) -- (tree-4-3) node [midway,above] {};
    \draw[->] (tree-5-2) -- (tree-6-3) node [midway,below] {};
    \draw[->] (tree-2-3) -- (tree-1-4) node [midway,below] {};
    \draw[->] (tree-2-3) -- (tree-2-4) node [midway,below] {};
    \draw[->] (tree-2-3) -- (tree-3-4) node [midway,below] {};
    \draw[->] (tree-2-3) -- (tree-4-4) node [midway,below] {};
    \draw[->] (tree-2-3) -- (tree-5-4) node [midway,below] {};
    \draw[->] (tree-4-3) -- (tree-2-4) node [midway,below] {};
    \draw[->] (tree-4-3) -- (tree-3-4) node [midway,below] {};
    \draw[->] (tree-4-3) -- (tree-4-4) node [midway,below] {};
    \draw[->] (tree-4-3) -- (tree-5-4) node [midway,below] {};
    \draw[->] (tree-4-3) -- (tree-6-4) node [midway,below] {};
    \draw[->] (tree-6-3) -- (tree-3-4) node [midway,below] {};
    \draw[->] (tree-6-3) -- (tree-4-4) node [midway,below] {};
    \draw[->] (tree-6-3) -- (tree-5-4) node [midway,below] {};
    \draw[->] (tree-6-3) -- (tree-6-4) node [midway,below] {};
    \draw[->] (tree-6-3) -- (tree-7-4) node [midway,below] {};
  \end{tikzpicture}



  \subsection{Rentabilités espérées du produit}

\noindent On peut avoir de même un tableau résumant les rentabilités actuarielles annuelles sous la forme suivante en considérant que le prix de départ est de la valeur liquidative de référence ( les x sont des cas impossibles) :

   \begin{tikzpicture}[>=stealth,sloped]
    \matrix (tree) [%
      matrix of nodes,
      minimum size=1cm,
      column sep=3cm,
      row sep=1cm,
    ]
    {
          &       &      & 4.06\%\\
          &       & 4.40\%  & 3.44\%\\
          & 4.50\% &      & 2.80\% \\
        0 &       & 2.23\% & 2.13\% \\
          & x   &      & 1.45\% \\
          &       & x  & 0.74\%\\
          &       &      & 0\% \\
    };
    \draw[->] (tree-4-1) -- (tree-3-2) node [midway,above] {};
    \draw[->] (tree-4-1) -- (tree-5-2) node [midway,below] {};
    \draw[->] (tree-3-2) -- (tree-2-3) node [midway,above] {};
    \draw[->] (tree-3-2) -- (tree-4-3) node [midway,below] {};
    \draw[->] (tree-5-2) -- (tree-4-3) node [midway,above] {};
    \draw[->] (tree-5-2) -- (tree-6-3) node [midway,below] {};
    \draw[->] (tree-2-3) -- (tree-1-4) node [midway,below] {};
    \draw[->] (tree-2-3) -- (tree-2-4) node [midway,below] {};
    \draw[->] (tree-2-3) -- (tree-3-4) node [midway,below] {};
    \draw[->] (tree-2-3) -- (tree-4-4) node [midway,below] {};
    \draw[->] (tree-2-3) -- (tree-5-4) node [midway,below] {};
    \draw[->] (tree-4-3) -- (tree-2-4) node [midway,below] {};
    \draw[->] (tree-4-3) -- (tree-3-4) node [midway,below] {};
    \draw[->] (tree-4-3) -- (tree-4-4) node [midway,below] {};
    \draw[->] (tree-4-3) -- (tree-5-4) node [midway,below] {};
    \draw[->] (tree-4-3) -- (tree-6-4) node [midway,below] {};
    \draw[->] (tree-6-3) -- (tree-3-4) node [midway,below] {};
    \draw[->] (tree-6-3) -- (tree-4-4) node [midway,below] {};
    \draw[->] (tree-6-3) -- (tree-5-4) node [midway,below] {};
    \draw[->] (tree-6-3) -- (tree-6-4) node [midway,below] {};
    \draw[->] (tree-6-3) -- (tree-7-4) node [midway,below] {};
  \end{tikzpicture}

  \subsection{Expression analytique des flux}
\noindent Soit 
\[ A = \{ \text{Dow Jones Euro Stoxx 50}, \text{Standard \& Poor's 500}, \text{Footsie 100}, \text{Nikkei 225} \} \]\\

\noindent et

\[ B = \{1,2,6\} \]\\
\\
\\
On cherche alors à déterminer le flux qui sera versé à la date t en fonction de la valeur liquidative de référence notée $N_{0}$.
On obtient alors pour $t \in \left\lbrace 1,2,3,4,5,6 \right\rbrace$ :

\begin{multline*}

  F_{versé}(t) = N_{0}*(1+ \sum_{i=1}^{t}(4.5\%*\mathds{1}_{\left\lbrace \sum_{I \in A} \mathds{1}_{\left\lbrace Perf_{I}(i)>10\% \right\rbrace}>2 \right\rbrace}) * (\mathds{1}_{\left\lbrace \sum_{I \in A} \mathds{1}_{\left\lbrace Perf_{I}(\min{(t,2)})>20\% \right\rbrace}>2 \right\rbrace}\\ 
  + \mathds{1}_{\mathds{1}_{t=6}}*\mathds{1}_{\left\lbrace \sum_{I \in A} \mathds{1}_{\left\lbrace Perf_{I}(1)>20\% \right\rbrace}<3 \right\rbrace}*\mathds{1}_{\left\lbrace \sum_{I \in A} \mathds{1}_{\left\lbrace Perf_{I}(2)>20\% \right\rbrace}<3 \right\rbrace})*\mathds{1}_{\{t \in B\}}) \\
  
\end{multline*}

\section{Clients visés}

Afin de mieux comprendre les risques et le produit, nous devons réfléchir aux types d'investisseurs concernés par un tel produit.\\

\indent \textbf{Le Montant minimum à la première souscription} est de 150 euros. Ce montant est faible comparé au montant minimum de souscription dans le cas d'obligations qui est généralement de l'ordre de quelques milliers d'euros, ou encore pour se créer un portefeuille d'actions diversifié. En effet, certaines actions peuvent atteindre plusieurs centaines d'euros l'unité. Ce montant de 150 euros pour une souscription permet à un particulier d'investir dans cet OPCVM.\\
\indent Les simulations sur les données historique de marché permettent de calculer des rendements fictifs, calculés en fonction des dates d'échéance passées de la formule. Elles permettent de visualiser le comportement de la formule lors des différentes phases de marché traversées au moments de simulations.\\

\begin{center}
\caption{Simulations sur des données historiques de marché}
\end{center}

\begin{center}
\includegraphics[scale=0.5]{simulations_historiques.png}
\end{center}

\indent Sur les simulations réalisées entre le 31 décembre 1991 et le 3 novembre 2009 on remarque que le rendement du FCP reste positif ou nul (hors commission de souscription) quelles que soit les phases de marché traversées. Néanmoins on remarque que le rendement des indices lors de hausse des marchés est très élevé par rapport au rendement de de la formule qui est en moyenne supérieur au taux sans risque. Lors de phase de volatilité des marchés, alors que les performances des indices n'obéissent pas à une tendance claire, le rendement du FCP reste positif ou nul et est supérieur au taux sans risque. Enfin pendant une période de chute des marchés, tandis que les indices ont des performances négatives, le rendement du FCP reste positif ou nul mais inférieur au taux sans risque. \\
\indent Ces simulations de scénarios montrent donc que la formule permet un rendement équivalent à un taux sans risque bien qu'il arrive lors de chute des marchés que le rendement du FCP soit nul. Lors de situation de marché stable où les indices sont élevés, les performances de la formule est nettement inférieur aux performances des indices. \\
\indent L'OPCVM permet donc à un particulier d'investir dans un actif comparable à un taux sans risque. Il permet en plus d'avoir un produit qui suit les performances des indices sur des dates d'échéances. Néanmoins le produit n'est pas comparable à un Tracker car globalement la formule n'engendre pas des rendements du même ordre de grandeur que les indices. Les institutions financières qui cherche à investir dans des actifs peu risqués utiliseront plutôt des obligations tandis que s'il cherchent à obtenir un rendement élevés qui suivent les performances des 4 indices, ils achèteront des Tracker. De plus Fonds est plafonné à 924 459 parts, de valeur nominale de 150€, ce qui fait un total de 138 668 850€. Ce montant est faible pour des investisseurs qui seraient des institutions financières.\\
\indent L'OPCVM se veut donc un produit pour des particuliers qui souhaitent obtenir un rendement similaire au taux sans risque avec une prise de risque faible. L'OPCVM garantit ainsi un rendement positif ou nul quelles que soient les fluctuations du marché. De plus un placement en OPCVM permet d'obtenir un portefeuille diversifié tout en restant accessible par son prix de souscription minimum faible et sécurisé grâce au cadre légal et réglementaire.\\

\newpage
\section{Analyse des risques du produit}

Dans cette partie, nous allons analyser tous les points qui peuvent constituer un risque pour la banque émettrice du produit puis pour le détenteur du produit. \\


\subsection{Risques pour la Banque}


\indent La banque est exposée au \textbf{risque de couverture}. La banque va devoir être en mesure de payer au porteur à l'une des dates d'échéances anticipées ou à la date d'échéance maximum, la valeur liquidative de référence, hors commission de souscription, majorée d'un gain final qui est acquis au cours du temps. Une fluctuation des marchés durant la durée de vie de la formule peut rendre moins efficace la couverture mise en place par la société de gestion. Le risque de couverture peut se matérialiser sous la forme d'autres risques: le risque de contrepartie, le risque sur la volatilité, risque de liquidité et le risque de taux. \\

\indent \textbf{Le risque de contrepartie} est le risque que ses clients soient dans l'incapacité de rembourser leurs emprunts, ou qu'une institution financière avec laquelle le fonds a des opérations en cours soit défaillante. Ce risque est présent car le portefeuille de l'OPCVM est constitué d'actifs qui entrent dans ces deux catégories.\\
\indent Pour illustrer ce risque, prenons un Swap de taux payeur (le taux fixe est payé, et le taux variable est perçu), conclu avec une contrepartie qui se finance intégralement à taux variable. Lorsque les taux augmentent, la rentabilité et donc la valeur de ce Swap de taux payeur augmente, cependant la qualité de crédit de la contrepartie qui se finance à taux variable baisse puisque le coût de son financement a augmenté. Dans cet exemple, le Swap de taux payeur est soumis au risque que l'exposition à la contrepartie soit inversement corrélée à la qualité de crédit de celle-ci. \\
\indent Une manière de se couvrir contre ce risque de contrepartie consiste à diversifier les Swaps avec différentes institutions financières. Plus les Swaps sont établis avec un nombre d'institutions financières différentes et plus les pertes engendrées par les Swaps sont réduites.\\
Une autre manière consiste à établir des CDS avec d'autres contreparties. Lors de la défaillance d'une contrepartie sur un Swap, le contrat permettra de se protéger contre la perte possible par le biais du vendeur de CDS qui se porte garant.  \\

\indent L'objectif de gestion du FCP est de permettre aux porteurs de recevoir aux dates d'échéances anticipées ou à la date d'échéance finale la valeur liquidative de référence avec le gain engendré durant la période. Si le portefeuille d'OPCVM est constitué d'actifs risqués alors \textbf{le risque de volatilité} augmenterait l'incertitude des rendements qu'engendrent les actifs et donc le rendement final garanti aux porteurs. \\
\indent C'est par exemple le cas de l'action d'une société plus endettée, ou disposant d'un potentiel de croissance plus fort et donc d'un cours plus élevé que la moyenne. Si la croissance des ventes est moins forte qu'espérée ou si l'entreprise peine à rembourser sa dette, la chute du cours sera très forte. \\
\indent Il est possible de tirer parti d'une volatilité élevée. Par exemple, prendre une position longue sur la volatilité peut être intéressant sur le plan tactique: la propension naturelle au retour vers la moyenne de la volatilité implicite offre des opportunités d'achat de couverture lorsque la volatilité retourne vers la moyenne après s'être hissée à un niveau élevé. \\
Les couvertures possibles se font grâce à des actifs financiers qui limitent l'achat maximale ou la vente minimale d'un actif. Des options exotiques comme les options Lookback permettent d'engendrer un gain important lorsque les marchés deviennent très volatiles. Les options Barrières permettent de limiter la prime payée sur ces options grâce à une volatilité élevée qui va enclencher la barrière et activer l'option. Les FRA, les Caps, les Floors et les Swaps vont permettre à la société de gestion de limiter le taux maximale pour l'emprunt (achat d'un FRA, Cap ou Swap payeur) ou limiter le taux minimale pour le prêt (achat d'un Floor). \\


\indent \textbf{Le risque de liquidité} implique qu'une position, dans le portefeuille de l'OPCVM, ne puisse être cédée, liquidée ou clôturée pour un coût limité et dans un délai suffisamment court, compromettant ainsi la capacité de l'OPCVM à se conformer aux dates d'échéances et aux dispositions qui sont prévus pour le client. Si les marchés sont peu liquides, l'efficacité de la couverture est diminuée car la banque rencontrera des difficultés à acheter ou vendre les actifs de couverture aux moments nécessaires. \\
\indent Par exemple, financer des crédits 10 ans par des emprunts à 3 mois représente un risque majeur en termes de liquidité. En effet, tous les trois mois, il faut trouver le refinancement et cela pendant 10 ans. En revanche, à l'inverse si on finance des crédits 5 ans par des emprunts 10 ans le risque de liquidité est minime. Pour se fixer des limites en liquidité, il faut donc se fixer un niveau de transformation sur chaque maturité. \\
\indent Le Cash at Risk est une mesure du risque de liquidité. Elle se mesure soit à partir de la comparaison des échéances contractuelles des dettes et des estimations des recettes de trésorerie, ou bien à travers un budget de trésorerie. Cet indicateur reprend globalement les modélisations issues des calculs de Value at Risk. Le calcul de cette mesure avant d'investir dans un actif va permettre de savoir l'exposition au risque de liquidité que prend l'institution. Cette mesure permet aussi d'obtenir des informations sur le risque de liquidité d'actifs déjà en possession. Une augmentation du Cash at Risk d'un produit peut permettre de liquider une position avant que la vente soit impossible. 
\\

\indent La variation des taux d'intérêts a un impact sur la variation du prix ou la valorisation d'un actif. La société de gestion possède donc un \textbf{risque de taux} sur l'ensemble de ses actifs. L'évolution des taux d'intérêts va générer un effet inverse sur le cours des obligations. \\
\indent Par exemple, le risque de taux se matérialise quand une banque refinançant un prêt à long terme à taux fixe par un emprunt à taux variable fait face à une hausse brutale des taux d'intérêts. Le risque est d'autant plus élevé que la maturité des actifs à taux fixe est éloignée et que la proportion d'actifs à taux fixe est importante dans le bilan de l'OPCVM.\\
\indent Pour se couvrir, on détermine par maturité la sensibilité du portefeuille au taux. On réduit cette sensibilité si on anticipe une hausse des taux et on augmente cette sensibilité si on anticipe une baisse. Plusieurs contrats permettent de se couvrir contre ce risque: les Swaps, les Caps ou les Collars.\\

\indent Pour un OPCVM de la zone euro, le risque de change, c'est-à-dire la variation des cours d'une devise -autre que l'euro - dans laquelle un fonds aurait investi, est présent. Le risque de change tient au fait que l'intérêt sur les obligations puis le remboursement du capital peut être payé dans une devise faible, ou qui se dévalue, et par conséquent le paiement des intérêts et le rendement final de l'investissement sera donc plus faible que celui attendu par l'investisseur et idem pour le capital.\\
\indent Playlist 2 est un OPCVM libellé en euros jouant sur les marchés internationaux et n'est pas donc pas exempt de ce risque. En effet la devise principale de l'investisseur est l'euro. Le portefeuille de l'OPCVM est donc valorisé en euro, néanmoins la devise des obligations dans lequel l'OPCVM investit peut être dans une monnaie différente comme le dollar qui peut aussi bien s'apprécier que se déprécier face à la monnaie européenne. \\
\indent Pour se couvrir contre le risque de change, on détermine une position par devise. On gère cette position en la couvrant via des actifs financiers comme des Swaps, des Caps ou des FRA sur taux ou encore des options Quanto en euro(similaire à une option classique sur un actif étranger où les flux récupéré lors de l'activation de l'option sont en euro) qui vont permettre de limiter ce risque de change. \\

\indent Un OPCVM possède une exposition au risque d'actions. Elle tient compte des opérations en cours et notamment de celles réalisées sur les marchés des dérivés, qui peuvent augmenter ou diminuer les risques de la gestion selon les fluctuations des marchés et du prix de l'action. \\
Avoir un seuil d'exposition minimum au risque d'actions de 60\% signifie que les gérants ne pourront à aucun moment réduire l'exposition en-deçà de ce seuil, de façon à maintenir une corrélation étroite avec l'évolution des marchés conforme à la nature de cette catégorie d'OPCVM.\\
Une couverture sur le risque d'action peut se faire par le biais de contrats d'achat ou de vente passés à l'avance grâce à des produits dérivés fermes (Futures ou Forwards) ou des produits dérivés optionnels (options).\\


\subsection{Risques pour le détenteur}

\indent En souscrivant au fonds, le détenteur, s'il garde ses parts, est assuré de récupérer au moins la Valeur Liquidative de Référence majorée du gain final. Cependant, cette valeur peut être sensibles à de nombreux risques présents sur les marchés et auxquels la banque doit se couvrir comme expliqué plus haut. Le détenteur n'est donc pas à l'abri de voir la valeur de son investissement diminuer au cours de sa participation dans la formule.\\

\indent Le détenteur du produit est exposé au risque financier systémique sur les 4 marchés de référence pour l'OPCVM. Une absence d'augmentation de la performance d'au moins 10\% des indices aux dates de constatations entraînera un gain final réduit.\\
Le porteur ne peut se couvrir contre cette absence d'augmentation des marchés.\\

\indent Le détenteur est exposé au risque que le fonds soit dissous ou soit fusionné. Dans ce cas, les détenteurs de parts pourraient voir sa valeur liquidative garanti réduite. En effet, en souscrivant à l'OPCVM, le porteur peut choisir qu'en cas de liquidation du fonds il est remboursé par la société garante en numéraire ou en titres. Dans le cas des titres, le risque que les titres reçus ne soient pas liquides sont grands car dans le cas d'une liquidation du fonds, la société de gestion n'a pas pu vendre ses actifs. Dans le cas du numéraire, le risque pour le porteur est que le garant ne possède pas assez de trésorerie pour rembourser tous les épargnants. \\
Pour se couvrir contre un tel risque, une étude de la répartition des remboursements est à voir. Généralement, investir beaucoup dans un OPCVM permet d'avoir une priorité sur le remboursement en cas de liquidation. \\

\indent Le client est également exposé à risque d'augmentation de l'inflation en effet malgré de potentielle gain il peut voir son pouvoir d'achat diminuer. \\
\indent En cas d'augmentation de l'inflation, lors du versement des liquidités par le FCP, les revenus perçus seront déprécié au cours du temps. La valeur liquidative de référence n'est donc pas exactement celle à laquelle le porteur aurait pu s'attendre.\\
\indent Une couverture contre l'inflation est par le biais d'un intermédiaire bancaire d'investir dans des produits indexés sur l'inflation. Les OAT et les Swaps d'inflation française permettent de se couvrir contre ce risque.\\



\subsection{Synthèse des risques}

\noindent Les risques énoncés précédemment peuvent donc être synthétisés selon le tableau suivant :\\

\begin{spacing}{1.2}
\begin{center}
\begin{tabular}{|c|c|c|}
  \hline
  Type de risque & Détenteur & Banque \\
  \hline
  Risque de valeur de la VLR & \checkmark &  \\
  \hline
  Risque de contrepartie & \checkmark &  \\
  \hline
  Risque de marché & \checkmark & \checkmark\\
  \hline
  Risque de couverture &  & \checkmark \\
  \hline
  Risque de volatilité &  & \checkmark \\
  \hline
  Risque de liquidité &  & \checkmark\\
  \hline
  Risque d'inflation & \checkmark & \\
  \hline 
  Risque de change &  & \checkmark \\
  \hline
  Risque d'actions &  & \checkmark\\
  \hline
  Risque de taux &  & \checkmark\\
  \hline
\end{tabular}
\end{center}
\end{spacing}
 

\section{Corrélation entre les indices}

Nous avons cherché à déterminer dans cette partie si il existait des corrélations entre les performances des différents indices en présence auquel cas, à partir de l'évolution d'un des indice on pourrait selon un certain intervalle de confiance prévoir l'évolution de la performance de certains autres indices parmi les 4 indices.  
C'est ainsi que nous avons commencé par calculer les différentes performances de chaque indice depuis le 29 avril 2010, puis que nous avons utilisé le logiciel R pour déterminer ces corrélations. 

\begin{spacing}{1.2}
\begin{center}
\begin{tabular}{|c|c|c|c|c|}
  \hline
   & Footsie 100 & Eurostoxx 50 & Nikkei 225 & S\& P 500 \\
  \hline
  Footsie 100 & 1 & 0.568 & 0.848 & 0.894\\
  Eurostoxx 50 & 0.568 & 1 & 0.507 & 0.278 \\
  Nikkei 225 & 0.848 & 0.507 & 1 & 0.822\\
  S \& P 500 & 0.894 & 0.278 & 0.822 & 1\\
  \hline
\end{tabular}
\end{center}
\end{spacing}

Afin d'observer au mieux les corrélations relevées ci dessus, nous avons décidé de tracer les performances de chaque indice jour par jour en prenant comme point de référence les valeurs des cours le 29/04/2010 et le 30/04/2010 pour le Nikkei 225 puisque celui ci n'était pas côté le 29/04/2010.
Aux vues des corrélations obtenues précédemment, nous devrions observé des rentabilités évoluant la plupart du temps dans le même sens et les mêmes proportions pour les couples (Footsie,Nikkei), (Footsie,S \& P) et finalement de manière un peu moindre entre (Nikkei, S \& P).


\begin{center}
\caption{Performances des différents indices du produit depuis le 29/10/2010}
\end{center}


\begin{center}
\includegraphics[scale=0.5]{Correlations_indices.jpg}
\end{center}

On constate suite à l'observation de ce graphique que l'évolution des performances du Footsie 100 et du S \& P 500 sont souvent dans les mêmes proportions et on observe les mêmes pics ainsi que les mêmes creux dans la plupart des cas comme par exemple au niveau du 30/06/2012.
Ceci offre donc de l'information à l'investisseur puisque celui-ci en anticipant une forte hausse du S \& P par exemple saura que le Footsie et le Nikkei toutes proportions gardées, devraient sentir une hausse sensiblement identique or comme les performances sont évalués sur 3 indices et que les meilleurs taux de rendements actuariels sont obtenus pour des échéances de un et deux ans. Ceci implique que l'investisseur pourra espérer obtenir le meilleur rendement actuariel possible avec ce produit.

\newpage

\section{Analyse logiciel}

\subsection{Architecture logiciel}

Voici une première schématisation de l'architecture de notre produit.

\begin{center}
\caption{Architecture}
\end{center}

\begin{center}
\includegraphics[scale=0.5]{graphe.jpg}
\end{center}

\noindent Ainsi on constate que l'architecture que nous avons adopté se base sur les cours disponibles sur des sites spécialisés tels que \lstinline!yahoo finance!, ainsi à partir d'un extracteur de données, nous récupérerons les données de ce site concernant les indices voulus, ce qui alimentera notre base de données. Les données seront alors traitées par le moteur de calcul notamment pour connaître la performance actuelle du fonds, avoir diverses modélisations de l'évolution des indices. Il vient alors l'interface qui à l'aide des sorties du moteur de calcul peut afficher l'évolution des différents indices ainsi que la performance du fonds.


\section{Annexes}

\subsection{Pseudo code pour la valeur du flux versé en t}


\begin{lstlisting}

public double flux(double ValNom, int temps){
	double flux = 0;
	double performance =0 ;
	int CountSupDix = 0;
	int CountSupVingtEnUn=0;
	int CountSupVingtEnDeux=0;
	int CountSupVingtEnMin=0;
	int borne = min(t,2);
	int verse = 0;
	if (temps==1 || temps==2 || temps==6){
		if (temps==6){
	 		foreach(indice in ListIndice){
				if (perf(Indice,1)>20%){
					CountSupVingtEnUn++;
				}
				if (perf(Indice,2)>20%){
					CountSupVingtEnDeux++;
				}
			}
			if (CountSupVingtEnUn < 3 && CountSupVingtEnDeux < 3 ){
				verse = 1;
			}
		}else{
			 foreach(indice in ListIndice){
				 if(perf(Indice,borne)>20%){
					 CountSupVingtEnMin++;
				 }
			 }
			 if (CountSupVingtEnMin>2){
				 verse =1;
			 }
		}
	}
	if (verse){
		for (int k=1; k<=t; k++){
			foreach(indice in ListIndice){
				if(perf(Indice,k)>10%){
					 CountSupDix++;
				 }
			}
			if (CountSupDix > 2){
				performance += 4.5%;
			 }
		}
		return ValNom*(1+performance);
	}
	return 0;
}
\end{lstlisting}


\end{document}

